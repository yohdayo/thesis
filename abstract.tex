\thispagestyle{empty}
\section*{要約}
近年,Siriなどのスマートフォン上で動作する音声エージェントやAlexaなどのスマートスピーカに代表されるように
音声対話技術が進展してきている.これは既に社会において音声を通じて対話するシステムが受け入れられて
活用されていると見ることができる.しかしながら,人の日常活動における対話は,上述のスマートスピーカとの
対話よりも複雑であり,現在の音声対話技術でも対話を継続して目的を達成することは困難である.
例えば,カウンターセールス(接客業務)では,対話の内容にに応じて,言葉だけでなく
視線や表情,身振り手振りなどのジェスチャを通じて対話相手の要求に応えていく.
こうした対話における言葉に付随する複数の要素を通じて,実際の人と対話できるシステムは
どのように構築すれば良いかの方針がまだ明らかになっていない.
そこで本研究ではカウンターセールスをタスクを対象とした音声対話システムを構築して
人との対話実験を行うことで構築したシステムの評価を行う.具体的にはセールスの内容として
旅行業務に焦点を置くとともに,対話だけではなく,視線や表情だけでなくジェスチャまで
コントロールできるアンドロイドを活用した音声対話システムを構築する.
評価実験として対話ロボットコンペティションに参加し,ショッピングモール内の雑音がある部屋で
人と対話する実験を行った.動作確認実験と共に対話体験者に対してアンケートによる評価を得たので報告する.

