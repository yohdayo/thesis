\section{関連研究}
\label{関連研究}
本章では,音声対話システムに関する先行研究について述べる.

\subsection{音声対話システムに関する先行研究}
\label{音声対話システムに関する先行研究}

対話システムは,数度のフェーズを経て,古いものでは1950年代後半から開発が行われている.\cite{higashinaka2020python}初期の対話システムとしてWeizenbaumのELIZA\cite{weizenbaum1966eliza}やWinogradのSHRDLU\cite{winograd1971shrdlu}が挙げられる.ELIZAは手作業で作成したIf-Thenルールに基づき挙動するシステムで,簡単な単語の一致によるパターンマッチングにより動作する.SHRDLUは対話により積み木を動かすというシステムで,「赤いブロックを左に移動して」という発言に対し,赤いブロックが2つある場合「どちらの赤いブロックですか?」等の問い返しをすることができる.これらの2つのシステムは限定的な状況でしか問題を解決することができず,現実の問題を解決するには至っていない.1980年代になると,データベースをもとにしたエキスパートシステムが台頭する.BuchananらのMYCIN\cite{buchanan1984rule}は医療診断システムとして専門医に若干劣るほどの診断を可能とした.BobrowらのGUS\cite{bobrow1977gus}はフレーム表現と呼ばれる知識構造を利用し,ユーザの発話から発話理解を行う.フレーム表現は今日の対話システムでも利用されることが多く,本実験でもフレーム表現を利用して構築を行う.エキスパートシステムのデータベース作成は全て手作業で行われるため,専門知識を体系的にシステムに落とし込む作業がボトルネックとなった.2000年代になると機械学習の発展により,事例から対話を生成することが可能となる.WuらのTOD-BERT\cite{wu2020tod}は10万件を超える対話事例から対話を自動生成する.機械学習により音声認識や音声合成などの精度も飛躍的に向上し,音声認識や音声合成を用いた対話システムの構築が積極的に行われている.\cite{higsshinakalive}

\subsection{対話システムの類型}
\label{対話システムの類型}
対話システムは以下のような観点で分類することができる.
\begin{description}
    \item[タスクの有無]
    \item[ユーザの人数]
    \item[モダリティ]
    \item[主導権]
    \item[身体性の有無]   
\end{description}
