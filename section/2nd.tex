\section{関連研究}
\label{関連研究}
本章では,音声対話システムに関する先行研究について述べる.

\subsection{音声対話システムに関する先行研究}
\label{音声対話システムに関する先行研究}

対話システムは,数度のフェーズを経て,古いものでは1950年代後半から開発が行われている.\cite{higashinaka2020python}初期の対話システムとしてWeizenbaumのELIZA\cite{weizenbaum1966eliza}やWinogradのSHRDLU\cite{winograd1971shrdlu}が挙げられる.ELIZAは手作業で作成したIf-Thenルールに基づき挙動するシステムで,簡単な単語の一致によるパターンマッチングにより動作する.SHRDLUは対話により積み木を動かすというシステムで,「赤いブロックを左に移動して」という発言に対し,赤いブロックが2つある場合「どちらの赤いブロックですか?」等の問い返しをすることができる.これらの2つのシステムは限定的な状況でしか問題を解決することができず,現実の問題を解決するには至っていない.1980年代になると,データベースをもとにしたエキスパートシステムが台頭する.BuchananらのMYCIN\cite{buchanan1984rule}は医療診断システムとして専門医に若干劣るほどの診断を可能とした.BobrowらのGUS\cite{bobrow1977gus}はフレーム表現と呼ばれる知識構造を利用し,ユーザの発話から発話理解を行う.フレーム表現は今日の対話システムでも利用されることが多く,本実験でもフレーム表現を利用して構築を行う.エキスパートシステムのデータベース作成は全て手作業で行われるため,専門知識を体系的にシステムに落とし込む作業がボトルネックとなった.2000年代になると機械学習の発展により,事例から対話を生成することが可能となる.WuらのTOD-BERT\cite{wu2020tod}は10万件を超える対話事例から対話を自動生成する.機械学習により音声認識や音声合成などの精度も飛躍的に向上し,音声認識や音声合成を用いた対話システムの構築が積極的に行われている.\cite{higsshinakalive}

\subsection{対話システムの類型}
\label{対話システムの類型}
前節で述べたように様々な対話システムが先行研究で提案されているが,対話システムは概ね下記のような観点で
分類することができる.
\begin{description}
    \item[(1)] タスクの有無
    \item[(2)] ユーザの人数
    \item[(3)] モダリティ
    \item[(4)] 主導権
    \item[(5)] 身体性の有無
\end{description}
である.まず\textbf{(1)}は対話システムが完遂すべきタスクを仮定して作られているかどうかである.上記の例ではMYCINが該当すると考えられる.また近年ではWebページ上でユーザの質問に答えるチャットボットが構築されているがこうしたシステムもタスクを仮定したシステムであるといえる.一方,タスクが明確に存在しないシステムとしてはELIZAが挙げられる.達成すべきタスクはないが上述の簡素な手法で人間との対話が続いたことが報告されている\cite{weizenbaum1966eliza}.
\textbf{(2)}は会話に参加するユーザ人数のことでほとんど上記に掲載するものは2名の想定であると考えられる.複数人発話はユーザが複数人存在し,対話ロボットもその中の一人として会話するもので発話のタイミングの研究などが成されている\cite{oki2012}.

\textbf{(3)}のモダリティとはテキストとしての言葉の発話以外の周辺情報を意味する.具体的には声の大きさや抑揚,身振りや手振りなども入る.こうしたマルチモーダル対話を行うシステムとして有名なのはPepperが挙げられる.また,本研究の対話ロボットもマルチモーダル対話システムの一種である.
\textbf{(4)}は対話の主導権をシステムが握るかユーザが握るかである.どちらか一方ではなく主導権が切り替わる場合は混合主導型の対話システムと呼ばれる\cite{higashinaka2020python}.最後に\textbf{(5)}は対話ロボットが身体を持つかどうかの分類である.Siriは身体を持たないがPepperは身体を有する.また本研究の対話ロボットも身体を持つことでバーチャルエージェントではなく実在するシステムとして発話相手と対話する.
