\section{考察}
\label{考察}
\subsection{対話ロボットに関する考察}
構築した対話システムについて基本的な動作に対する評価と体験者による評価について考察する
基本的な動作確認における対話において構築した対話ロボットは的確に回答してタスクを終えている.
よって,雑音環境下にもかかわらず,通常の会話に対して実時間で全処理を行えたことは構築が成功していると
考えられる.これは本対話システムの対話制御はノートパソコンで処理を行ったが,他の音声認識や
発話などはAPIで外部サーバが処理を行っているため負荷が分散されて実行される.こうした近年の
通信速度の向上も,対話ロボットの実現に影響を与えている.

今回,アンドロイドによるジェスチャーも加えた対話システムを構築したが,アンドロイド制御部分には
まだまだ課題があることが明らかになった.まずアンドロイド自身には物理的な制約がいくつかありその
範囲内での構築に限られる点である.例えば表情を笑顔など変形する際に,長時間の表情の変形は
材質が伸びてしまう可能性があることから,大きな変化などは制限されている.また,アンドロイド制御には
シミュレータも利用したが,外観は実際の動作が確認する以外方法がない.ATRの所有物であり,また動作には
技術者の調整が必要であるため,自由に試すことができず当初の考えを実施することはできなかった.
この点はアンドロイドというバーチャルではない要素を研究する場合のボトルネックになることが明らかになった.

一方で,アンドロイドの存在は大きく,ショッピングモールで体験者をその場で募集したが希望者が
複数得られたのは単なる携帯デバイス上で会話する見慣れたエージェントと異なるためであったと考えられる.
身体を持つロボットとの対話という研究テーマは明らかになっていない要素が多く,評価方法も含めて
これからさらに実施して問題を整理する必要があると考えられる.

\subsection{対話管理に関する考察}
対話管理として対話の流れを決める上で背景となるシナリオを設定し,フレームを利用した発話理解から対話履歴を
参考にしながら発話するシステムを構築した.シナリオで重視したことはユーザに対話の主導権をある程度
与えることであった.これは,今回のタスクが2件の旅行先から選択するという限定した内容であるため,
ユーザに自由度を与える事で,対話が楽しいものになることを期待した部分と,システムとしてどの程度
自由な発話を受け入れられるかを確認することが目的であった.

よって実験中に想定したシナリオにない発話があり,システムが意図を獲得できずに同じ回答を繰り返すだけの
対話になることがあった.こうしたことは想定の範囲ではあったが人間の発話の自由度を確認するデータ
を獲得できた.また,一方で,コンペティションに参加した他のシステムのうち,上位に入賞したシステム
の発表から,他のシステムでも想定外の発話があることがわかった.しかしながら,評価の高いシステムは
ユーザの発話意図が取れなかった場合,当たり障りのない発話を返して,対話システムが意図が取れていない
ことが伝わらないようにする工夫が成されていた.こうした点での構築は行ってこなかったため評価が他の
システムに比べて低いと考えられる.また,コンペティションに参加したことで,他のシステムの構築方法
の成功している部分を確認することができた.



\subsection{実際の体験者に対する対話から得られたこと}
対話コンペティションが実施された期間はCovid-19の影響により,「ららぽーとEXPOCITY」における入場客が元々の
想定よりも少なくなってしまい,コンペティションにおける対話の体験者が集まりが少なくなった.そのためコンペティション
における各システムの評価ではビデオ評価も実施されたが,元々のカウンターセールスを実施する対話ロボットを
生成するという目的からは,少ないながらも実際にショッピングモールに来ていた客に対する対話の状況は
対話ロボットはどう実際の現場で利用されるかという部分を確認することができた.大きな点として2つある.
1つ目は破綻した場合の体験者の対話と2つ目は対話管理シナリオである.

まず1つ目の破綻した場合の体験者の対話について記述する.対話が破綻した場合,構築した対話ロボットでは
質問が理解できないことを話者に対して伝える会話が繰り返されることがあった.その場合に,体験者の方は
ゆっくり話すなど,音声認識を意識して聞き取りやすくなるような話し方で対話ロボットに会話した.しかし
ながら音声認識装置は,普段の会話スピードで学習していることから通常よりも遅く発話された場合には
認識精度が低下する可能性があり,対話の成立がますます困難になることが予測される.こうした対話破綻は
起こることが想定されるので,対話ロボットの音声認識では,体験者がゆっくり発話した場合も識別できる
モードを設定することが必要であることが考えられる.

2つ目は対話管理シナリオについてである.今回,本研究で作成した対話ロボットはフレーム表現を利用して,
背景となる対話管理シナリオを想定して次の行動を規定した.よって,背景となる対話管理から逸脱した
会話を求められた場合,会話が破綻することがあった.逸脱した場合の事例で興味深いのは観光案内先
における想定できないような質問である.例えば,箕面大滝に対して「猿と遊べますか」といった内容の
質問があった.こうした質問には対話管理シナリオとして単に事実を伝えるというよりは楽しさや面白さ
といった要素を含めた返答が求められる.こうした面白さの会話はSiriなどに見られるが実際に
必要であることが実験で明らかになった.
現在のシステムでは構築は難しいが,ヒューマノイドに対してより普通では聞かない内容を聞くことが
あることを対話管理シナリオとして用意しておくことは必要であると考えられる.



