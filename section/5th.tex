\section{考察}
\label{考察}
\subsection{対話ロボットに関する考察}
構築した対話システムについて基本的な動作に対する評価と体験者による評価について考察する
基本的な動作確認における対話において構築した対話ロボットは的確に回答して会話を終了した.


\subsection{対話制御に関する考察}

\subsection{実際の体験者に対する対話から得られたこと}
対話コンペティションが実施された期間はCovid-19の影響により,「ららぽーとEXPOCITY」における入場客が元々の
想定よりも少なくなってしまい,コンペティションにおける対話の体験者が集まりが少なくなった.そのためコンペティション
における各システムの評価ではビデオ評価も実施されたが,元々のカウンターセールスを実施する対話ロボットを
生成するという目的からは,少ないながらも実際にショッピングモールに来ていた客に対する対話の状況は
対話ロボットはどう実際の現場で利用されるかという部分を確認することができた.大きな点として2つある.
1つ目は破綻した場合の体験者の対話と2つ目は対話管理シナリオである.

まず1つ目の破綻した場合の体験者の対話について記述する.対話が破綻した場合,構築した対話ロボットでは
質問が理解できないことを話者に対して伝える会話が繰り返されることがあった.その場合に,体験者の方は
ゆっくり話すなど,音声認識を意識して聞き取りやすくなるような話し方で対話ロボットに会話した.しかし
ながら音声認識装置は,普段の会話スピードで学習していることから通常よりも遅く発話された場合には
認識精度が低下する可能性があり,対話の成立がますます困難になることが予測される.こうした対話破綻は
起こることが想定されるので,対話ロボットの音声認識では,体験者がゆっくり発話した場合も識別できる
モードを設定することが必要であることが考えられる.

2つ目は対話管理シナリオについてである.今回,本研究で作成した対話ロボットはフレーム表現を利用して,
背景となる対話管理シナリオを想定して次の行動を規定した.よって,背景となる対話管理から逸脱した
会話を求められた場合,会話が破綻することがあった.逸脱した場合の事例で興味深いのは観光案内先
における想定できないような質問である.例えば,箕面大滝に対して「猿と遊べますか」といった内容の
質問があった.こうした質問には対話管理シナリオとして単に事実を伝えるというよりは楽しさや面白さ
といった要素を含めた返答が求められる.こうした面白さの会話はSiriなどに見られるが実際に
必要であることが実験で明らかになった.
現在のシステムでは構築は難しいが,ヒューマノイドに対してより普通では聞かない内容を聞くことが
あることを対話管理シナリオとして用意しておくことは必要であると考えられる.



