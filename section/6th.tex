\section{まとめ}
\label{まとめ}
本研究ではカウンターセールスをタスクを対象とした音声対話システムを構築して
人との対話実験を行うことで構築したシステムの評価を行った.具体的にはセールスの内容として
旅行業務に焦点を置くとともに,対話だけではなく,視線や表情だけでなくジェスチャまで
コントロールできるアンドロイドを活用した音声対話システムを構築した.
音声認識ではGooogleのSpeech-toTextを利用し,音声合成にはAmazonのPollyを利用した.
また口形状の生成には Oculus Lipsync Unityを利用した.
アンドロイドはATRが管理するアンドロイドを利用した.
アンドロイドの動作として首の動作,姿勢,表情を対話に合わせて制御した.
対話管理の中心的なモジュールである発話理解としてフレーム表現を用いて,発話を構造化して保持し,
次の対話の応答の際に,過去の発話履歴を利用する機構を構築した.
評価実験として対話ロボットコンペティションに参加するために旅行会話に合わせた発話行為タイプ
を整理して実装した.評価実験ではショッピングモール内に設置された雑音がある部屋で,
人と対話する実験を行った.動作確認実験では意図通りの会話を実行することができた.
また実験の2日間で合計21人と会話することができた.アンケートによる評価結果では
コンペティション内の他のシステムと比較して対話の満足度および対応の好ましさで
全12台中の8番目という評価を得た.しかしながら一方で,対話の破綻が起こることがあり
情報の十分さなどで高い評価が得られなかった.実験結果から自然な対話を一部提供できたが
対話管理シナリオで想定しない会話の場合の対応がよくないことが明らかになった.
%この点については今後の課題である.

